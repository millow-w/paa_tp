% \usepackage[english]{babel}
\usepackage[brazilian]{babel}
\usepackage[letterpaper,top=2cm,bottom=2cm,left=3cm,right=3cm,marginparwidth=1.75cm]{geometry}

% Useful packages
\usepackage{amsmath}
\usepackage{graphicx}
\usepackage{hyperref}
\usepackage{listings}
\usepackage[figure,table,lstlisting]{totalcount}
\usepackage{tcolorbox}
\usepackage{listings}
\usepackage{xcolor}
\usepackage{booktabs}
\usepackage{fontspec}
\setmainfont{Arial}

\hypersetup{
    colorlinks=true,
    linkcolor=black,
    filecolor=magenta,      
    urlcolor=cyan,
    pdftitle={TP - Projeto e Análise de Algoritmos},
    pdfpagemode=FullScreen,
}

\definecolor{verde}{rgb}{0.25,0.5,0.35}
\definecolor{jpurple}{rgb}{0.5,0,0.35}

\lstset{
  language=C++,
  basicstyle=\ttfamily\small,
  keywordstyle=\color{jpurple}\bfseries,
  stringstyle=\color{red},
  commentstyle=\color{verde},
  morecomment=[s][\color{blue}]{/**}{*/},
  extendedchars=true,
  showspaces=false,
  showstringspaces=false,
  numbers=left,
  numberstyle=\tiny,
  breaklines=true,
  backgroundcolor=\color{cyan!10},
  breakautoindent=true,
  captionpos=b,
  xleftmargin=0pt,
  tabsize=4
}

\renewcommand{\lstlistingname}{Código}
\renewcommand{\lstlistlistingname}{Lista de Códigos Fonte}

\newcommand\conditionalLoF{%
    \iftotalfigures
        \listoffigures
    \fi}
\newcommand\conditionalLoT{%
    \iftotaltables
        \listoftables
    \fi}
\newcommand\conditionalLoL{%
    \iftotallstlistings
        \lstlistoflistings
    \fi}

\newcommand{\DESCRICAO}[1]{
    \setlength{\spaceskip}{0.5em plus 1em minus 0.1em}%
    \ifdim\lastskip>0pt \hspace{0.5em plus 0.5em minus 0.1em}\fi
    \texttt{\textbf{\color{red}#1}}
}

\newcommand{\CAPA}[4]{
  \begin{titlepage}

    \vfill

    \begin{center}
      \large Universidade Federal de Ouro Preto - UFOP
    \end{center}

    \begin{center}
      \large Instituto de Ciências Exatas e Biológicas - ICEB
    \end{center}

    \begin{center}
      \large Departamento de Computação - DECOM
    \end{center}

    \begin{center}
      \large Ciência da Computação
    \end{center}

    \vfill

    \begin{center}
      {\Huge #1}\\[0.5cm]
      {\Large #2}
    \end{center}

    \vfill

    \begin{center}
      \large #3
    \end{center}

    \begin{center}
      \large Professor: #4
    \end{center}

    \vfill

    \begin{center}
      \large Ouro Preto\\
      \today
    \end{center}

  \end{titlepage}
}
